\documentclass{article}

\usepackage{amsmath}
\usepackage{graphicx}

\begin{document}
\title{MCMD P5 -- Simulation of biomolecules}
\author{Alex Arash Sand Kalaee\\ \texttt{kalaee@teorfys.lu.se}}
\maketitle

\textbf{An introduction to molecular dynamics simulations with AMBER}

This introduction tutorial focused on getting familiar with AMBER.
The SUMMARY files depicted a system heated to 300 K and remained stable while the total energy increased in tandem with the temperature. Also, we studied the RMSD with CPPTRAJ to how it changed over time.

\textbf{Using VMD with AMBER}

Another introduction about simulations. We learned how to visualise a molecule and make a movie.

\textbf{Loop dynamics of the HIV-1 integrase core domain}

Actual simulations. Via the input file the molecule was added to
a water bath and ions added. We minimized the energy and
equilibrated the system through heating. By looking at the RMSD
of the trajectory we tracked the system backwards to determine when
the system was sufficiently equilibrated.

\textbf{A case study in folding Trp-Cage}

We follow the steps in an article to study how the 
Trp-Cage folds. The final folding of the protein yielded a
higher energy state the one obtained in the article. Possibly due to shorter simulation time.

\textbf{Thermodynamic Integration using soft core potentials}

Calculation of the free energy difference between two systems.
We find $\Delta G =1.83$.

\textbf{Computing binding enthalpy values}

$\Delta H = H_\mathrm{complex} + H_\mathrm{water} + H_\mathrm{host}+H_\mathrm{guest} = -21.4 $.

\textbf{An introduction to CPPTRAJ}

Learned how to load a topology and trajectory, and how to do the
actions on the loaded system.

\textbf{RMSD Analysis in CPPTRAJ}

RMSD to references with the same type of topology and RMSD with different topology were presented.

\textbf{Hydrogen Bond Analysis with CPPTRAJ}

We find that RMSD change and changes in the number of solute-solute and solute-solvent bonds are negatively correlated.


Finally, I did \textbf{Protonating a protein with MAESTRO}.
\end{document}
